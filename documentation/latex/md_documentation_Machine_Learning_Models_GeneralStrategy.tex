To date, the C\+O\+V\+I\+D-\/19 pandemic has more or less affected the whole world, {\bfseries{both socially and economically}}. Thus, the {\bfseries{driving motivation}} behind A\+MI and our project is to investigate its {\bfseries{respective impact}} on various domains such as online traffic and economic performance. To this, we have decided for the following ML model strategy. ~\newline


First, data shall be collected that reflects {\bfseries{both pre-\/ and post-\/corona}} outbreak statistics in several {\bfseries{everyday scenarios}} such as streaming activity, etc. The underlying data sets are then {\bfseries{split up}} accordingly into two different time frames\+: pre and post corona-\/start.

The former data set is used to train an ML model which is able to sufficiently {\bfseries{characterize and predict}} the model outcome for the {\bfseries{respective domain of interest}} (internet traffic, social media, porn consumption, etc.) for a scenario where the {\bfseries{corona pandemic did not occur}}. By doing this, we can very clearly address the {\bfseries{impact of corona}} through the respective {\bfseries{error}} between the second post corona-\/start data set and the predicted model output. This interpretable error metric can then be effectively {\bfseries{visualized}} in the final web-\/interface.

Based on this model, its outcome and the respective error metric, a {\bfseries{future possible task}} -\/ that we will not further investigate -\/ could be to {\bfseries{fully capture the pandemic dynamic}} in a second subsequent model architecture. To this, the later data set can be used to train a model that allows for a {\bfseries{prediction of the actual outcome for the current corona pandemic}}. Assuming that the model has been sufficiently trained, it is in theory also able to predict a -\/ more or less realistic -\/ outcome for {\bfseries{e.\+g. worse case numbers, higher infection rates etc.}} The variation in this $\ast$$\ast$\char`\"{}corona harshness degree\char`\"{}$\ast$$\ast$ could be implemented into the web-\/interface alongside above first model output. Furthermore, it allows the user to {\bfseries{interact with the model}} by adjusting above dynamic parameters such as the case numbers. However, it is to be noted that this approach -\/ while being very interesting -\/ becomes an {\bfseries{increasingly speculative estimation}} for a variation of the dynamic parameters and especially so w.\+r.\+t. {\bfseries{long term interpretations}}. ~\newline


In summary, {\bfseries{two model approaches}} are proposed where the first predicts a future {\bfseries{without corona}} and the second one {\bfseries{with it}}. In order to be able to precisely {\bfseries{highlight the impact of the pandemic}} on the selected domains of our everyday lives, we decide to {\bfseries{realize the first model}} in this project whose predicted output can then be compared to the accumulated real-\/world pandemic data in the form of an {\bfseries{error metric}}. An example of how a {\bfseries{potential model output}} would compare against the real data is depicted in below figure for the streaming domain which has seen a {\bfseries{surge during the pandemic}} and especially after the lockdown.



The corresponding Matlab Plot Script as well as above .jpg can be found \href{/documentation/Matlab Plots/ML Strategy}{\texttt{ here}} 